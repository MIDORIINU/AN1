
A continuación se muestran los gráficos para la serie de tiempos para los valores iniciales $x_{1}(0) = 5$, $x_{2}(0) = 5$, $x_{3}(0) = 30$ con $t \in [0; 50]$.


\begin{figure}[!h] %htb
\begin{center}
\includegraphics[width=0.45 \linewidth,trim=4.5cm 1.5cm 3.5cm 0.5cm, clip, keepaspectratio=true, angle=0]{img/grafico_time_series_x1_1.png} %% 
\caption{\label{fig:fig_ts_x1_iv1}\footnotesize{Gráfico de la serie de tiempo para $x_{1}$ con $x_{1}(0) = 5$.}}
\end{center}
\end{figure}

% \clearpage

\begin{figure}[!h] %htb
\begin{center}
\includegraphics[width=0.45 \linewidth,trim=4.5cm 1.5cm 3.5cm 0.5cm, clip, keepaspectratio=true, angle=0]{img/grafico_time_series_x2_1.png} %% 
\caption{\label{fig:fig_ts_x2_iv1}\footnotesize{Gráfico de la serie de tiempo para $x_{2}$ con $x_{1}(0) = 5$.}}
\end{center}
\end{figure}

% \clearpage

\begin{figure}[!h] %htb
\begin{center}
\includegraphics[width=0.45 \linewidth,trim=4.5cm 1.5cm 3.5cm 0.5cm, clip, keepaspectratio=true, angle=0]{img/grafico_time_series_x3_1.png} %% 
\caption{\label{fig:fig_ts_x3_iv1}\footnotesize{Gráfico de la serie de tiempo para $x_{3}$ con $x_{1}(0) = 5$.}}
\end{center}
\end{figure}

\clearpage


\subsection{Observación del comportamiento}

El comportamiento asintótico de las soluciones es oscilante, sin embargo no se discierne una periocidad en las mismas. Las oscilaciones observadas parecen ser entre valores que se repiten a lo largo del intervalo, siendo distintos para cada serie de tiempo. 