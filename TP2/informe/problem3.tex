A continuación se muestran los gráficos para la serie de tiempos para los valores iniciales $x_{2}(0) = 5$, $x_{3}(0) = 30$ con $t \in [0; 50]$, pero ahora variando el valor incial de la primera función solución en los valores $x_{1}(0) = 5.01$, $x_{1}(0) = 5.001$ y $x_{1}(0) = 5.0001$.


\clearpage

\subsection{Series de tiempo para $x_{1}$}

%%%%%% x1

\begin{figure}[!h] %htb
\begin{center}
\includegraphics[width=0.5 \linewidth,trim=4.5cm 1.5cm 3.5cm 0.5cm, clip, keepaspectratio=true, angle=0]{img/grafico_time_series_x1_2.png} %% 
\caption{\label{fig:fig_ts_x1_iv2}\footnotesize{Gráfico de la serie de tiempo para $x_{1}$ con $x_{1}(0) = 5.01$.}}
\end{center}
\end{figure}

% \clearpage

\begin{figure}[!h] %htb
\begin{center}
\includegraphics[width=0.5 \linewidth,trim=4.5cm 1.5cm 3.5cm 0.5cm, clip, keepaspectratio=true, angle=0]{img/grafico_time_series_x1_3.png} %% 
\caption{\label{fig:fig_ts_x1_iv3}\footnotesize{Gráfico de la serie de tiempo para $x_{1}$ con $x_{1}(0) = 5.001$.}}
\end{center}
\end{figure}

% \clearpage

\begin{figure}[!h] %htb
\begin{center}
\includegraphics[width=0.5 \linewidth,trim=4.5cm 1.5cm 3.5cm 0.5cm, clip, keepaspectratio=true, angle=0]{img/grafico_time_series_x1_4.png} %% 
\caption{\label{fig:fig_ts_x1_iv4}\footnotesize{Gráfico de la serie de tiempo para $x_{1}$ con $x_{1}(0) = 5.0001$.}}
\end{center}
\end{figure}

\clearpage

%%%%%%%%%%%%%%%%%%%%%%%%%%%%


\subsection{Series de tiempo para $x_{2}$}

%%%%%% x2

\begin{figure}[!h] %htb
\begin{center}
\includegraphics[width=0.5 \linewidth,trim=4.5cm 1.5cm 3.5cm 0.5cm, clip, keepaspectratio=true, angle=0]{img/grafico_time_series_x2_2.png} %% 
\caption{\label{fig:fig_ts_x2_iv2}\footnotesize{Gráfico de la serie de tiempo para $x_{2}$ con $x_{1}(0) = 5.01$.}}
\end{center}
\end{figure}

% \clearpage

\begin{figure}[!h] %htb
\begin{center}
\includegraphics[width=0.5 \linewidth,trim=4.5cm 1.5cm 3.5cm 0.5cm, clip, keepaspectratio=true, angle=0]{img/grafico_time_series_x2_3.png} %% 
\caption{\label{fig:fig_ts_x2_iv3}\footnotesize{Gráfico de la serie de tiempo para $x_{2}$ con $x_{1}(0) = 5.001$.}}
\end{center}
\end{figure}

% \clearpage

\begin{figure}[!h] %htb
\begin{center}
\includegraphics[width=0.5 \linewidth,trim=4.5cm 1.5cm 3.5cm 0.5cm, clip, keepaspectratio=true, angle=0]{img/grafico_time_series_x2_4.png} %% 
\caption{\label{fig:fig_ts_x2_iv4}\footnotesize{Gráfico de la serie de tiempo para $x_{2}$ con $x_{1}(0) = 5.0001$.}}
\end{center}
\end{figure}

\clearpage

%%%%%%%%%%%%%%%%%%%%%%%%%%%%


\subsection{Series de tiempo para $x_{3}$}

%%%%%% x3

\begin{figure}[!h] %htb
\begin{center}
\includegraphics[width=0.5 \linewidth,trim=4.5cm 1.5cm 3.5cm 0.5cm, clip, keepaspectratio=true, angle=0]{img/grafico_time_series_x3_2.png} %% 
\caption{\label{fig:fig_ts_x3_iv2}\footnotesize{Gráfico de la serie de tiempo para $x_{3}$ con $x_{1}(0) = 5.01$.}}
\end{center}
\end{figure}

% \clearpage

\begin{figure}[!h] %htb
\begin{center}
\includegraphics[width=0.5 \linewidth,trim=4.5cm 1.5cm 3.5cm 0.5cm, clip, keepaspectratio=true, angle=0]{img/grafico_time_series_x3_3.png} %% 
\caption{\label{fig:fig_ts_x3_iv3}\footnotesize{Gráfico de la serie de tiempo para $x_{3}$ con $x_{1}(0) = 5.001$.}}
\end{center}
\end{figure}

% \clearpage

\begin{figure}[!h] %htb
\begin{center}
\includegraphics[width=0.5 \linewidth,trim=4.5cm 1.5cm 3.5cm 0.5cm, clip, keepaspectratio=true, angle=0]{img/grafico_time_series_x3_4.png} %% 
\caption{\label{fig:fig_ts_x3_iv4}\footnotesize{Gráfico de la serie de tiempo para $x_{3}$ con $x_{1}(0) = 5.0001$.}}
\end{center}
\end{figure}

\clearpage

%%%%%%%%%%%%%%%%%%%%%%%%%%%%

\subsection{Observación del comportamiento}

Se puede observar que la serie de tiempos es muy sensible a las variaciones de las condiciones inicales, con solo cambiar en muy poco solo uno de los valores inciales se observa que a pesar de que las series de tiempo siguen siendo oscilantes, los valores de las oscilaciones y  su exacta forma es completamente distinta en cada caso.