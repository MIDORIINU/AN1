
El script resuelve en principio los casos pedidos en el enunciado, y luego toma interactivamente con diálogos nuevos parámetros al usuario, los cuales valida y utiliza para resolver el nuevo sistema, en cada caso se grafica las funciones de posición (angulo) y su derivada (velocidad angular) y se calcula la integral del módulo de la posición (angulo) en el intervalo.\\


A continuación se listan los archivos de \textbf{MATLAB} y su función: \\\\


\textbf{\quotemarks{tp2.m}}~(apéndice~\apendref{apendix:file_tp2}): Script principal que se debe ejecutar para realizar todos los cálculos y generar los archivos de resultados.\\

\textbf{\quotemarks{pendulum.m}}~(apéndice~\apendref{apendix:file_pendulum}): Función que invoca nuestra implementación de Runge-Kutta de orden 4 para el sistema del péndulo.\\

\textbf{\quotemarks{rk4.m}}~(apéndice~\apendref{apendix:file_rk4}): Función con nuestra implementación del algoritmo Runge-Kutta de orden 4 en forma genérica matricial.\\

\textbf{\quotemarks{plot\_solution.m}}~(apéndice~\apendref{apendix:file_pts}): Función que grafica el ángulo y la velocidad obtenidas.\\

\textbf{\quotemarks{trapezcomp\_rk4.m}}~(apéndice~\apendref{apendix:file_trapezcomp_rk4}): Función que implementa el método de integración de trapecios compuesto adaptado a usar Runge-Kutta 4 (usado en Romberg).\\

\textbf{\quotemarks{romberg\_rk4.m}}~(apéndice~\apendref{apendix:file_romberg_rk4}): Función que implementa el método de integración de Romberg adaptado a usar Runge-Kutta de orden 4.\\

\textbf{\quotemarks{f\_rk4\_num\_sol.m}}~(apéndice~\apendref{apendix:file_f_rk4_num_sol}): Función auxiliar que se usa para generar un handle a función que lleva implicito el sistema de ecuaciones diferenciales a resolver por Runge Kutta de orden 4, se usa en el método adaptado de Romberg.\\




En el apéndice correspondiente~(apéndice~\apendref{apendix:files}) se incluye el código completo de cada archivo.\\\\

\textbf{\quotemarks{salida.txt}}~(apéndice~\apendref{apendix:file_salida}): La salida del script principal, se captura automáticamente en \textbf{MATLAB} al ejecutar el script principal.\\\\


El método de Romberg se implementó de manera de llamar a Runge-Kutta de orden 4, se optimizó de modo que solo se hace una llamada, para lograr esto se aprovechó el hecho de que el método de trapecios compuesto que se llama en cada nivel de Romberg, tiene un paso que es la mitad que en el nivel anterior, por lo tanto solo se resuelve el sistema para el paso mas pequeño, que corresponde al último nivel, luego en cada nivel se llama a una versión adaptada de trapecios que saltea los puntos necesarios para el paso necesario (se saltean $2^{(p - k)}$ donde $p$ es el nivel de Romberg, y $k$ corresponde al nivel en que se está calculando).

El método de Romberg se implementó,así como está descripto en la bibliografía, usando interpolación de Richardson y el método de trapecios compuesto, con las modificaciones antes mencionadas.\\

El error del método de integración de Romberg ($R_{n,n}$), asumiendo que la función es suficientemente diferenciable, está en $O \left(  h^{\left( 2 \cdot p  \right)}  \right)$, donde $h$ es el paso, que en el caso de esté método, vale en cada nivel $h = \frac{1}{2^{n}} \cdot \left( b - a \right)$, siendo $a$ y $b$ los límites de integración y $n$ el nivel, con lo que se tiene un error que está en $O \left( \frac{1}{2^{p}} \cdot \left( b - a \right)  \right)$ para el método de nivel $p$.\\

El nivel de Romberg usado fue de $15$, así que se espera un error del orden de $\frac{1}{2^{p}} \cdot \left( b - a \right) = \frac{20 - 0}{2^{30}} = 2 \times 10^{-8}$.




\clearpage

\subsection{Sobre los archivos de \textbf{MATLAB} y \textbf{Octave}}

Hay algunas cosas a comentar sobre las diferencias entre \textbf{MATLAB} y \textbf{Octave}, como se comentó anteriormente, se logró la compatibilidad de ejecución entre los entornos, sin embargo las salidas no son completamente equivalentes, debido a limitaciones en \textbf{Octave}, las salidas gráficas no son completamente equivalentes, en particular \textbf{MATLAB} permite la generación de \textbf{DataTips}, cosa que \textbf{Octave} aún no soporta, otra cuestión quizás mas importante es la eficiencia en ejecución, en algunos casos los tiempo de ejecución en \textbf{Octave} se hacen demasiado largos si se usan arrays muy extensos, lo cual se mitigó usando compilación condicional. Otra cosa a mencionar que no hace a la funcionalidad directamente, pero si a la presentación, es que debido a limitaciones en ambos entornos respecto a la codificación de los archivos y soporte incompleto o inadecuado de \textbf{UNICODE} en la línea de comando de \textbf{Windows}, se producen problemas en las salidas con símbolos que no sean parte de Latin-1 (ISO 8859-1), en particular las palabras con tilde, esto se ve aún mas complicado porque \textbf{Windows} usa una variante (CP1252) que no es completamente compatible con Latin-1 y el hecho de que los entornos de \textbf{MATLAB} y \textbf{Octave} no se comportan consistentemente en \textbf{Windows} y sistemas tipo Unix como \textbf{Linux}, en Unix es prácticamente universal la codificación de \textbf{UNICODE}, \textbf{UTF-8}. \textbf{MATLAB} sigue la codificación del sistema operativo, mientras que \textbf{Octave} intenta usar \textbf{UTF-8} siempre, pero en \textbf{Windows} no es completo el soporte. El tema es complicado y no hace al trabajo práctico el lidiar con el mismo, dado que trabajamos mayormente con \textbf{MATLAB}, tanto en \textbf{Windows} como en \textbf{Linux}, y la mayoría usa \textbf{Windows}, se optó por dejar los archivos en \textbf{CP1252}, siendo esta la codificación usual. Se incluyen simplemente por comodidad dos scripts de \textbf{Python}, \textbf{\quotemarks{utf8.py}} y \textbf{\quotemarks{cp1252.py}}, que convierten la codificación de todos los archivos \textbf{\quotemarks{.m}} a las respectivas codificaciones, de esa manera, según el sistema en que se ejecuten los scripts,  se puede lograr una salida con codificación correcta.


\clearpage