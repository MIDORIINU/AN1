
\subsection{Resumen enunciado}


%\bfseries{\color{red} Enunciado:}

\normalfont

Este TP consiste en realizar una implementación del algoritmo Runge-Kutta de orden 4, para cuyo testeo se propuso resolver el sistema de ecuaciones diferenciales que se obtiene a partir de la conocida ecuación diferencial de segundo orden que se obtiene al plantear la oscilación de un péndulo:


\begin{equation}
\ddot{\theta} + \frac{b}{m} \cdot \dot{\theta} + \frac{g}{l} \cdot sin \left( \theta \right) = 0
\label{eq:pendulum}
\end{equation}


Planteando:
\begin{equation*}
                \begin{array}{ll}
                  \theta = x_{1} \\
                  \dot{\theta} = x_{2} \\                
                \end{array}
\end{equation*}

Y reemplazando en la ecuación~\eqref{eq:pendulum}, se obtiene el sistema:

\begin{equation}
             \mathcolorbox{EQColor}{
             \left\{
                \begin{array}{ll}
                  \dot{x_{1}} = x_{1} \\
                  \dot{x_{2}} = - \frac{b}{m} \cdot x_{2} - \frac{g}{l} \cdot sin \left( x_{1} \right) \\
                \end{array}
              \right. 
              }
\end{equation}

Además de lo anterior, se pide que se implemente el método de integración de Romberg, y se aplique al cálculo del área encerrada bajo la curva del módulo del desplazamiento del péndulo.


\subsection{Resolución}

Para la resolución de la parte de programación del trabajo práctico e implementar los algoritmos pedidos, decidimos usar \textbf{MATLAB}, mayormente por conocerlo previamente y la sencillez con la que se pueden escribir scripts que implementen los algoritmos. A pesar de que la resolución se realizó en \textbf{MATLAB}, se prestó atención a la compatibilidad con \textbf{Octave}, ya que la compatibilidad en los paquetes básicos es alta y con un poco de cuidado y algo de programación condicional se puede lograr que los scripts funcionen en ambos entornos.
Todos los resultados numéricos se guardaron por código desde \textbf{MATLAB} en formato \textbf{\quotemarks{CSV}} y las imágenes se guardaron también por código en formato \textbf{\quotemarks{PNG}} y luego se incorporaron desde \LaTeX, para los archivos \textbf{\quotemarks{CSV}} se usó el paquete de \LaTeX \textbf{pgfplotstable}. 


