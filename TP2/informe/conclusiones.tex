
El método numérico implementado en el caso propuesto se comporta en general como es esperado, sirvió la comparación con los métodos proveídos por \textbf{MATLAB} u \textbf{Octave} y también tener un problema planteado para el cual la solución se conoce de antemano. El algoritmo de Runge-Kutta fue implementado en forma matricial, lo cual a pesar de ser mas complicado de pensar inicialmente, permite lograr un código mas simple y entendible.