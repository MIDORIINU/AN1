
%*********************************************
\subsection{Consideraciones para el código}
\thispagestyle{codeconsstyle}
\renewcommand{\themark}{Consideraciones para el código}
El código está escrito en \textbf{MATLAB}, se trato de hacerlo ordenado y con todos los comentarios necesarios, así como también se hizo un uso consistente del indentado con tabulaciones a 4 espacios. La presentación del código en el informe se hizo directamente desde los fuentes hacia \LaTeX\space usando el package \textbf{\quotemarks{listingsutf8}}~\cite{The_ListingsUTF8_Package}, que es una extensión con soporte para \textbf{UTF8} del paquete \textbf{\quotemarks{listings}}~\cite{The_Listings_Package}, este paquete produce una salida formateada y con coloreado del código y también permite el agregado de números de líneas, el código fuente en \textbf{MATLAB} es soportado directamente, la salida que se produce es muy buena, pero no es perfecta, ya que cuestiones como el tabulado o el ancho total de las líneas pueden producir problemas, en caso de que alguno de estos problemas hagan confuso o incomprensible el código por favor remitirse a los fuentes originales. 
 

\clearpage
%*********************************************


%******************************************************************************************
\lstset{language=MATLAB,numbers=left,xleftmargin=1em,stepnumber=1
,extendedchars=true,
literate=
{á}{{\'a}}1
{à}{{\`a}}1
{ã}{{\~a}}1
{é}{{\'e}}1
{ê}{{\^e}}1
{í}{{\'i}}1
{ó}{{\'o}}1
{õ}{{\~o}}1
{ú}{{\'u}}1
{ü}{{\"u}}1
{ç}{{\c{c}}}1
{Á}{{\'A}}1
{À}{{\`A}}1
{Ã}{{\~A}}1
{É}{{\'E}}1
{ê}{{\^E}}1
{Í}{{\'I}}1
{Ó}{{\'O}}1
{Õ}{{\~O}}1
{Ú}{{\'U}}1
{Ü}{{\"U}}1
}


\lstset{showspaces=false}
\lstset{showstringspaces=false}

\lstset{backgroundcolor=\color{white},rulecolor=\color{blue}}
\lstset{basicstyle=\ttfamily\color{Black}}

\lstset{keywordstyle=[1]\ttfamily\color{MATLABKeyword}\bfseries}
\lstset{keywordstyle=[2]\ttfamily\color{MATLABKeyword}}
\lstset{keywordstyle=[3]\ttfamily\bfseries\color{MATLABKeyword}}
\lstset{keywordstyle=[4]\ttfamily\bfseries\color{MATLABKeyword}}

\lstset{identifierstyle=\ttfamily\color{black}}
\lstset{commentstyle=\ttfamily\color{MATLABComment}\textit}
\lstset{stringstyle=\ttfamily\color{MATLABString}\upshape}
\lstset{tabsize=4}

\lstset{numberstyle=\ttfamily\color{Orange}\upshape}
\lstset{numbersep=5pt}

%\lstset{inputencoding=utf8/latin1}
\lstset{inputencoding=cp1252}



\fontencoding{T1}
\fontseries{m}
\fontsize{9pt}{10pt}
\selectfont





%******************************************************************************************


\subsection{Archivos fuente de \textbf{MATLAB}}
\label{apendix:files}
%*********************************************
\subsubsection{tp2.m}
\label{apendix:file_tp2}
%\renewcommand{\filename}{tp2.m}
\lstinputlisting{../code/tp2.m}
\clearpage
%*********************************************

%*********************************************
\subsubsection{pendulum.m}
\label{apendix:file_pendulum}
%\renewcommand{\filename}{pendulum.m}
\lstinputlisting{../code/pendulum.m}
\clearpage
%*********************************************

%*********************************************
\subsubsection{rk2.m}
\label{apendix:file_rk4}
%\renewcommand{\filename}{rk4.m}
\lstinputlisting{../code/rk4.m}
\clearpage
%*********************************************

%*********************************************
\subsubsection{plot\_solution.m}
\label{apendix:file_pts}
%\renewcommand{\filename}{plot_solution.m}
\lstinputlisting{../code/plot_solution.m}
\clearpage
%*********************************************

%*********************************************
\subsubsection{romberg.m}
\label{apendix:file_romberg}
%\renewcommand{\filename}{romberg.m}
\lstinputlisting{../code/romberg.m}
\clearpage
%*********************************************

%*********************************************
\subsubsection{trapezcomp.m}
\label{apendix:file_trapezcomp}
%\renewcommand{\filename}{trapezcomp.m}
\lstinputlisting{../code/trapezcomp.m}
\clearpage
%*********************************************

\normalfont
\normalsize

%---






